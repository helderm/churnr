\documentclass{kththesis}

% remove this if you are using XeLaTeX or LuaLaTeX
\usepackage[utf8]{inputenc}

% Use natbib abbreviated bibliography style
\usepackage[square,numbers]{natbib}
\bibliographystyle{unsrtnat}

\usepackage{lipsum} % This is just to get some nonsense text in this template, can be safely removed

\title{Sequential User Retention Modelling}
\alttitle{Detta är den svenska översättningen av titeln}
\author{Helder Martins}
\email{helder@kth.se}
\supervisor{Hedvig Kjellström (KTH) and Sahar Asadi (Spotify)}
\examiner{Patric Jensfelt}
\programme{Master in Machine Learning}
\school{School of Computer Science and Communication}
\date{\today}


\begin{document}

% Title page
\flyleaf

\begin{abstract}
  English abstract goes here.
  \lipsum[1-2]
\end{abstract}

\clearpage

\begin{otherlanguage}{swedish}
  \begin{abstract}
    Träutensilierna i ett tryckeri äro ingalunda en faktor där
    trevnadens ordningens och ekonomiens upprätthållande, och dock är
    det icke sällan som sorgliga erfarenheter göras ordningens och
    ekon och miens därmed upprätthållande. Träutensilierna i ett
    tryckeri äro ingalunda en oviktig faktor, för trevnadens
    ordningens och och dock är det icke sällan.
  \end{abstract}
\end{otherlanguage}

\cleardoublepage

\tableofcontents


% This is where the actual contents of the thesis starts
\mainmatter


\chapter{Introduction}

    Service providers, especially those that operates on the web, have experienced a rapid increase in their user base in the last few years. However, the number of companies in each market increased in a similar fashion, and maintaining your current users within your service grew in importance since the process of acquiring new users is costly and difficult to execute properly. One of the important steps towards that goal is to predict accurately what is the chance of a user leaving the service provider at any given time with the intent of acting before it happens, a problem which is called \emph{churn prediction}. Several classical techniques for predicting churn like Decision Trees and Logistic Regression have been used on recent work \cite{Lazarov2007}, however most of them make use of user behaviour data at a single point in time, normally the most recent one when the dataset was created. 
    
    Intuitively one can think that latent factors hidden in the temporal axis of the user behaviour data could yield a better prediction accuracy when comparing to a model which leverages a single and static point in time. For instance, a user that is gradually reducing its consumption of the service over time can be easily thought of a prospect churner. While this intuition is trivial to come up with, we are also interested in learning strong correlations between temporal aspects of the data and the churn rate which are not as clear to the eye. Our hypothesis is that making use of the properties hidden on user behaviour data over time, a churn rate predictor could improve its accuracy greatly when compared to methods which are static on time.
    
    A churn rate predictor which can identify possible churners accurately is just a trigger on the user retention process that can span several different stages. One of these stages is to identify what actions can be taken by the service provider as to avoid the user abandoning the service. An important problem to solve beforehand is to identify which features of the data have a strong statistical correlation with the churn rate. With that information, the providers could for example set up automated actions as to influence the value on these features. Learning which are the most important features is also of relevance for the task of choosing what data to use for training the predictor models, since commonly service providers have a vast amount of data about the user but only a fraction of that is of importance for predicting churn: training models with a full dataset will commonly introduce error on the system and take a large amount of computing resources to train.
    
    The goal of this project will be to research different models for churn prediction that can make use of the temporal aspect of user behaviour data at its fullest. Our hypothesis is that latent factors hidden on usage patterns may increase the accuracy of state-of-art predictors which considers only the state of the data at a single point in time. We shall experiment on different models that are known to perform well on time series data and compare the winning one against a baseline model which does not leverage the time property. Also in this project we shall research different algorithms for feature selection, and investigate which subset correlates strongly with the churn rate on a specific dataset.
    
    The user data and the required computational resources shall be provided by Spotify AB, a well known music streaming service, which joined this project in a tutoring partnership with the student and the university. Creating a suitable dataset for training our proposed models is also part of the project, and shall be done by first exploring the data at our disposal and identifying the features that highly correlates with churn. An example of what this dataset may contain is the user listening habits, usage patterns, user profile, and so on.

\section{Research Question}
        
	Can latent factors hidden on user behaviour data improve the accuracy of a churn predictor model when compared to the ones which does not leverage this property? Can techniques proven to be successful for time series prediction on different domains be used interchangeably for prediction of churn rate? What feature selection methods can most successfully extract the correlations between the feature value and churn rate? Can we reasonably interpret what was learnt by our predictor models as to allow the service provider to create actions to keep user retention level high? Can we identify different types of users which exerts an influence on the churn rate of the service?

\chapter{Related Work}

\textbf{Pudipeddi, Akoglu and Tong} \citep{Pudipeddi2014} In "User Churn in Focused Question Answering Sites: Characterization and Prediction", Q\&A sites like Stack Overflow were the focus of a study on user behaviour characterization and churn rate prediction. An extensive data exploration was made as to correlate features to the chance of a user leaving a service, and with those insights classical modelling techniques were used, were the best performing one was a decision tree. The approach used for extracting and categorizing features (temporal, frequency, gratitude, etc) and the insights that follow the study (like the importance of temporal features for predicting churn) is of high value and can be mapped to concepts on a different domain like a music streaming service with minor modifications.

\textbf{Wangperawong, Brun, Laudy and Pavasuthipaisit} \citep{Wangperawong2016} In "Churn analysis using deep convolutional neural networks and autoencoders" users were represented as 2-dimensional images where columns are tracked features and rows are days. The intuition behind this approach is that by using similar techniques successful on the Image Recognition domain, a similar positive result for the prediction of the churn rate could also be achieved. Two different Convolutional Neural Network models were used, and both outperformed a baseline Decision Tree model. An autoencoder was also used as to discover which features influences churn rate the most, and suggestions where made as to how improve the retention of users on the service.

\textbf{Tax and Verenich} \citep{Tax2016} In "Predictive Business Process Monitoring with LSTM Neural Networks" a technique was presented as to predict the next event and its timestamp on a running case of a business process (a help desk process, for example). The problem definition was formally explained, and three different LSTM architectures were experimented on. With these architectures, three problems were tackled: estimating the next activity and its timestamp, all the remaining activities in a use case and the remaining time of a process. This technique could be used for churn prediction by interpreting the user interaction with the service as actions, and "churn" could be an event that may or may not exist in the process chain.

\textbf{Runge et al.} \citep{Runge2014} In "Churn prediction for high-value players in casual social games, high-value players on casual social games were the focus of a study on churn analysis and prediction. Formal definitions for active players and churning were made, and the problem was defined as a binary classification task where users were labelled as leaving the service or not in the following week of when the models were trained. Four different models were then trained, and a neural network classifier obtained the best area under the curve (AUC) score, with logistic regression as a close second. An attempt was made as to include temporal data in the neural net by enhancing it with features learned with a hidden Markov model (HMM), however the accuracy was decreased due to parameter overfitting. The formal definitions of "churn" and "activity" can be applied to a music streaming service in a similar way, as the evaluation of performance using a series of receiver operating characteristic (ROC) curves. A detailed A/B test with real users was also performed, but a similar approach falls out of the scope of this project.

\textbf{Dror et al.} \citep{Dror2012} In "Churn prediction in new users of Yahoo! answers", an explorative study was made on the Yahoo! answers website as to discover an efficient churn predictor model and also the features that correlates with the user leaving the service, however differently from other works this paper focused on new users with less than a week of activity on the service. Features were grouped into Question, Answer and Gratification categories, and were used for training several different classifiers. For this dataset, random forests performed the best with logistic regression once again a close second. Features were also ordered by the amount of information gain that they provide, and the number of questions and answers are the top features on that regard (inversely correlating with churn), followed by the period of time the user is active and gratification features for answers given and questions made. Similarly to \citep{Pudipeddi2014}, the features exploration section could be mapped to a music streaming service in a similar way (eg. question/answers are sessions, gratitude is the user explicit feedback on recommended content), and the evaluation methods with ROC curves and information gain tables are also of interest. 

\chapter{Methods}

\lipsum

\bibliography{references}

\appendix

\chapter{Unnecessary Appended Material}

\end{document}
